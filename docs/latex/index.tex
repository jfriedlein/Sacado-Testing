Example usage of Sacado for tensor calculus\begin{DoxyAuthor}{Author}
jfriedlein
\end{DoxyAuthor}
\hypertarget{index_intro}{}\section{Introduction}\label{index_intro}
The way we see and use Sacado here is as follows\+: ~\newline
If you usually compute the following equation ~\newline
 \[ c = a + b \] for instance with data types double as ~\newline
 \[ 1.0 + 1.0 \rightarrow 2.0 \] your results is just a double number $ c $ that contains the value $ 2 $. ~\newline
 Using Sacado, on the other hand, the variable $ c_{fad} $ is now of, for example, data type 
\begin{DoxyCode}
Sacado::Fad::DFad<double> c\_fad;
\end{DoxyCode}
 As a result, $ c_{fad} $ now contains not just the number $ 2 $, but also all the derivatives of $ c_{fad} $ with respect to the previously defined degrees of freedom (set via command $\ast$.diff($\ast$)). ~\newline
The following figure tries to visualize this\+:  \begin{DoxyRefDesc}{Todo}
\item[\hyperlink{todo__todo000001}{Todo}]update this figure or add another one for second derivatives 

add another less general figure with c, a and b and explain what is meant by point p 

explain how to use the Wrapper (download the file \hyperlink{Sacado__Wrapper_8h}{Sacado\+\_\+\+Wrapper.\+h}, \hyperlink{CMakeCache_8txt_a986ccfc90e04633694fe6cff5472be19}{include}, ...) 

Check if factor 0.\+5 is also necessary for d\+\_\+sigma / d\+\_\+phi\end{DoxyRefDesc}


If you right away want to use Sacado, then you might skip the first examples and jump to Ex3B. There we show how to use the \char`\"{}\+Sacado\+\_\+\+Wrapper\char`\"{} that does everything from Ex2 and Ex3 in just a view lines of code. This does not mean that the here shown approach is the fastest or most efficient, it is just simple and easy to use.

Furthermore, if you, for instance, compute problems with two-\/fields (e.\+g. displacement and scalar damage) and you need tangents with respect to both a tensor (e.\+g. strain tensor) and a scalar (e.\+g. damage variable), you can use the \hyperlink{namespaceSacado__Wrapper}{Sacado\+\_\+\+Wrapper} as shown in Ex4.

Some more basics\+: ~\newline
One can access the double value of $ c_{fad} $ with the Sacado command $\ast$.val()\+: 
\begin{DoxyCode}
\textcolor{keywordtype}{double} c\_value = c\_fad.val();
\end{DoxyCode}
 The derivatives of $ c_{fad} $ can be accessed with the command $\ast$.dx()\+: 
\begin{DoxyCode}
\textcolor{keywordtype}{double} d\_c\_d\_a = c\_fad.dx(0);
\textcolor{keywordtype}{double} c\_c\_d\_b = c\_fad.dx(1);
\end{DoxyCode}
 The arguments of {\itshape dx}, namely 0 and 1 are the numbers corresponding to the dof that belong to {\itshape a} and {\itshape b}. More details on how to set this up and use it, are given in example Ex1.

Some resources/links\+: ~\newline
You can use Sacado to compute general derivatives of functions (with or without tensors) with respect to variables (double, Tensors, ...). \begin{DoxyRefDesc}{Todo}
\item[\hyperlink{todo__todo000002}{Todo}]link the Sacado and D\+II pages\end{DoxyRefDesc}


The here shown examples shall solely show how Sacado can be applied and give some background and a look under the hood. The code is neither elegant nor efficient, but it works. A more user-\/friendly version is provided by means of the \char`\"{}\+Sacado\+\_\+\+Wrapper\char`\"{}. ~\newline
\begin{DoxyRefDesc}{Todo}
\item[\hyperlink{todo__todo000003}{Todo}]add list of files and an overview \end{DoxyRefDesc}


\begin{DoxyNote}{Note}
This documentation and code only protocol my first steps with Sacado. They are not guaranteed to be correct neither are they verified. Any comments, criticism, corrections, feedback, improvements, ... are appreciated and very well welcomed.
\end{DoxyNote}
\hypertarget{index_code}{}\section{The commented program}\label{index_code}

\begin{DoxyCode}
\textcolor{comment}{/*}
\textcolor{comment}{ * Author: jfriedlein, 2019}
\textcolor{comment}{ *      dsoldner, 2019}
\textcolor{comment}{ */}
\end{DoxyCode}
 \hypertarget{index_includes}{}\section{Include Files}\label{index_includes}
The data type Symmetric\+Tensor and some related operations, such as trace, symmetrize, deviator, ... for tensor calculus 
\begin{DoxyCode}
\textcolor{preprocessor}{#include <deal.II/base/symmetric\_tensor.h>}
\end{DoxyCode}
 C++ headers (some basics, standard stuff) 
\begin{DoxyCode}
\textcolor{preprocessor}{#include <iostream>}
\textcolor{preprocessor}{#include <fstream>}
\textcolor{preprocessor}{#include <cmath>}
\end{DoxyCode}
 Sacado (from Trilinos, data types, operations, ...) 
\begin{DoxyCode}
\textcolor{preprocessor}{#include <Sacado.hpp>}
 
\textcolor{preprocessor}{#include "\hyperlink{Sacado__Wrapper_8h}{Sacado\_Wrapper.h}"}
\end{DoxyCode}
 Those headers are related to data types and autodiff, but don\textquotesingle{}t seem to be needed 
\begin{DoxyCode}
\textcolor{comment}{//#  include <deal.II/base/numbers.h>}
\textcolor{comment}{//#  include <deal.II/differentiation/ad/ad\_number\_traits.h>}
\textcolor{comment}{//#  include <deal.II/differentiation/ad/sacado\_number\_types.h>}
\end{DoxyCode}
 According to the basics of deal.\+ii-\/programming (see dealii.\+org and \href{https://www.dealii.org/current/doxygen/deal.II/step_1.html}{\tt https\+://www.\+dealii.\+org/current/doxygen/deal.\+I\+I/step\+\_\+1.\+html} for a start) 
\begin{DoxyCode}
\textcolor{keyword}{using namespace }\hyperlink{namespacedealii}{dealii};
\end{DoxyCode}
 Defining a data type for the Sacado variables (here we simply used the standard types from the deal.\+ii step-\/33 tutorial\textquotesingle{}s introduction) 
\begin{DoxyCode}
\textcolor{keyword}{using} \hyperlink{Sacado__example_8cc_a868b94676739e612d9c95940e70892a9}{fad\_double} = Sacado::Fad::DFad<double>;   \textcolor{comment}{// this data type now represents a double, but
       also contains the derivative of this variable with respect to the defined dofs (set via command *.diff(*))}
\end{DoxyCode}
 \hypertarget{index_Ex1}{}\section{1. example\+: simple scalar equation}\label{index_Ex1}

\begin{DoxyEnumerate}
\item example\+: simple scalar equation from deal.\+ii-\/tutorial step-\/33 (see the introduction there to get a first impression, \href{https://www.dealii.org/current/doxygen/deal.II/step_33.html}{\tt https\+://www.\+dealii.\+org/current/doxygen/deal.\+I\+I/step\+\_\+33.\+html}) \begin{DoxyRefDesc}{Todo}
\item[\hyperlink{todo__todo000004}{Todo}]clean up the documentation of the classes\end{DoxyRefDesc}

\end{DoxyEnumerate}


\begin{DoxyCode}
\textcolor{keywordtype}{void} \hyperlink{Sacado__example_8cc_a71b2675e62203edc430e7ffc8a365193}{sacado\_test\_scalar} ()
\{
    std::cout << \textcolor{stringliteral}{"Scalar Test:"} << std::endl;
\end{DoxyCode}
 define the variables used in the computation (inputs\+: a, b; output\+: c; auxiliaries\+: $\ast$) as the Sacado-\/data type 
\begin{DoxyCode}
\hyperlink{Sacado__example_8cc_a868b94676739e612d9c95940e70892a9}{fad\_double} a,b,\hyperlink{CMakeCache_8txt_aac1d6a1710812201527c735f7c6afbaa}{c};
\end{DoxyCode}
 initialize the input variables a and b; This (a,b) = (1,2) will be the point where the derivatives are computed. Compare\+: y=x² -\/$>$ (dy/dx)(@x=1) = 2. We can only compute the derivative numerically at a certain point. 
\begin{DoxyCode}
 a = 1;
 b = 2;

a.diff(0,2);  \textcolor{comment}{// Set a to be dof 0, in a 2-dof system.}
b.diff(1,2);  \textcolor{comment}{// Set b to be dof 1, in a 2-dof system.}
\end{DoxyCode}
 Our equation here is very simply. But you can use nested equations and many standard mathematical operations, such as sqrt, pow, sin, ... 
\begin{DoxyCode}
c = 2*a + std::cos(a*b);
\textcolor{keywordtype}{double} *derivs = &c.fastAccessDx(0); \textcolor{comment}{// Access the derivatives of}
\end{DoxyCode}
 Output the derivatives of c with respect to the two above defined degrees of freedom (dof) 
\begin{DoxyCode}
    std::cout << \textcolor{stringliteral}{"Derivatives at the point ("} << a << \textcolor{stringliteral}{","} << b << \textcolor{stringliteral}{")"} << std::endl;
    std::cout << \textcolor{stringliteral}{"dc/da = "} << derivs[0] << \textcolor{stringliteral}{", dc/db="} << derivs[1] << std::endl;
\}
\end{DoxyCode}
 \hypertarget{index_Ex2}{}\section{2. example\+: Preparation for the use of Sacado with tensors}\label{index_Ex2}
Here we want to introduce tensors for the first time. Hence, we limit ourselves to a trivial equation relating the strain tensor {\itshape eps} with dim x dim components with the stress tensor {\itshape sigma}. Both here used tensors are symmetric, hence we use the Symmetric\+Tensor class and have to keep some details in mind (see below factor 0.\+5 related to Voigt-\/\+Notation). Don\textquotesingle{}t be scared by the enormous number of repetitive lines of code, everything shown in this example and the following will be handled by the \hyperlink{namespaceSacado__Wrapper}{Sacado\+\_\+\+Wrapper} with roughly four lines of code. 
\begin{DoxyCode}
\textcolor{comment}{/*}
\textcolor{comment}{ * 2. example: use of tensors}
\textcolor{comment}{ */}
\textcolor{keywordtype}{void} \hyperlink{Sacado__example_8cc_a8ef4ff1e9526ca8451cdcd1678366d2c}{sacado\_test\_2} ()
\{
    std::cout << \textcolor{stringliteral}{"Test 2:"} << std::endl;
\end{DoxyCode}
 First we set the dimension {\itshape dim\+:} 2\+D-\/$>$dim=2; 3\+D-\/$>$dim=3 ~\newline
 This defines the \char`\"{}size\char`\"{} of the tensors and the number of dofs. Ex2 only works in 3D, whereas the following Ex3 is set up dimension-\/independent. 
\begin{DoxyCode}
\textcolor{keyword}{const} \textcolor{keywordtype}{unsigned} \textcolor{keywordtype}{int} dim = 3;
\end{DoxyCode}
 Declare our input, auxiliary and output variables as Symmetric\+Tensors consisting of fad\+\_\+doubles (instead of the standard Symmetric\+Tensor out of doubles) 
\begin{DoxyCode}
SymmetricTensor<2,dim, fad\_double> sigma, eps;
\end{DoxyCode}
 Init the strain tensor (the point at which the derivative shall be computed) 
\begin{DoxyCode}
eps[0][0] = 1;
eps[1][1] = 2;
eps[2][2] = 3;
eps[0][1] = 4;
eps[0][2] = 5;
eps[1][2] = 6;
\end{DoxyCode}
 Now we declare the dofs. The derivative to a tensor requires all components, therefore we set the components of the strain tensor here one by one as the dofs. Because our tensors are symmetric, we only need 6 components in 3D instead of 9 for a full second order tensor 
\begin{DoxyCode}
eps[0][0].diff(0,6);
eps[1][1].diff(1,6);
eps[2][2].diff(2,6);
eps[0][1].diff(3,6);
eps[0][2].diff(4,6);
eps[1][2].diff(5,6);
\end{DoxyCode}
 The equation describing the stresses (here just a simple test case) 
\begin{DoxyCode}
sigma = eps;
\end{DoxyCode}
 Let\textquotesingle{}s output the computed stress tensor. 
\begin{DoxyCode}
std::cout << sigma << std::endl;
\end{DoxyCode}
 The resulting values of {\itshape sigma} are fairly boring, due to our simple equation. It is the additional output generated by this, that is interesting here\+: ~\newline
output\+: ~\newline
1 \mbox{[} 1 0 0 0 0 0 \mbox{]} 4 \mbox{[} 0 0 0 1 0 0 \mbox{]} 5 \mbox{[} 0 0 0 0 1 0 \mbox{]} 4 \mbox{[} 0 0 0 1 0 0 \mbox{]} 2 \mbox{[} 0 1 0 0 0 0 \mbox{]} 6 \mbox{[} 0 0 0 0 0 1 \mbox{]} 5 \mbox{[} 0 0 0 0 1 0 \mbox{]} 6 \mbox{[} 0 0 0 0 0 1 \mbox{]} 3 \mbox{[} 0 0 1 0 0 0 \mbox{]} ~\newline
The numbers 1, 4, 5, 4, ... are the entries in the stress tensor {\itshape sigma}. In square brackets we see the derivatives of sigma with respect to all the dofs set previously given in the order we defined them above. Meaning\+: The first entry in the square brackets corresponds to the 0-\/th dof set by 
\begin{DoxyCode}
eps[0][0].diff(0,6); 
\end{DoxyCode}
 referring to the component (0,0) in the strain tensor {\itshape eps}.

Computing the derivatives for certain components of the resulting tangent modulus\+: ~\newline
We now access these lists of derivatives (output above in square brackets) for one component of the stress tensor {\itshape sigma} at a time. 
\begin{DoxyCode}
\{
\end{DoxyCode}
 Access the derivatives corresponding to the component (0,0) of the stress tensor {\itshape sigma} 
\begin{DoxyCode}
\textcolor{keywordtype}{double} *derivs = &sigma[0][0].fastAccessDx(0);
\end{DoxyCode}
 The following output will show us the same derivatives that we already saw above, just formatted differently ~\newline
output\+: d\+\_\+sigma\mbox{[}0\mbox{]}\mbox{[}0\mbox{]}/d\+\_\+eps = 1 , 0 , 0 , 0 , 0 , 0 , 
\begin{DoxyCode}
    std::cout << \textcolor{stringliteral}{"d\_sigma[0][0]/d\_eps = "};
    \textcolor{keywordflow}{for} ( \textcolor{keywordtype}{unsigned} \textcolor{keywordtype}{int} i=0; i<6; ++i)
        std::cout << derivs[i] << \textcolor{stringliteral}{" , "};
    std::cout << std::endl;
\}
\{
\end{DoxyCode}
 Access the derivatives corresponding to the component (1,2) of the stress tensor {\itshape sigma} 
\begin{DoxyCode}
\textcolor{keywordtype}{double} *derivs = &sigma[1][2].fastAccessDx(0);
\end{DoxyCode}
 output\+: d\+\_\+sigma\mbox{[}1\mbox{]}\mbox{[}2\mbox{]}/d\+\_\+eps = 0 , 0 , 0 , 0 , 0 , 1 , 
\begin{DoxyCode}
        std::cout << \textcolor{stringliteral}{"d\_sigma[1][2]/d\_eps = "};
        \textcolor{keywordflow}{for} ( \textcolor{keywordtype}{unsigned} \textcolor{keywordtype}{int} i=0; i<6; ++i)
            std::cout << derivs[i] << \textcolor{stringliteral}{" , "};
        std::cout << std::endl;
    \}
\}
\end{DoxyCode}
 \hypertarget{index_Ex3}{}\section{3. example\+: Using a slightly more complicated stress equation}\label{index_Ex3}

\begin{DoxyCode}
\textcolor{keywordtype}{void} \hyperlink{Sacado__example_8cc_ae45e1df0eec246dbb6f2c3d28a2a58e4}{sacado\_test\_3} ()
\{
    std::cout << \textcolor{stringliteral}{"Test 3:"} << std::endl;
 
    \textcolor{keyword}{const} \textcolor{keywordtype}{unsigned} \textcolor{keywordtype}{int} dim = 3;
\end{DoxyCode}
 Here we also define some constant, for instance the bulk modulus {\itshape kappa} and the second Lamè parameter {\itshape mu}. We now also define one of our constants as fad\+\_\+double. By doing this we can use the normal multiplication (see below). 
\begin{DoxyCode}
\textcolor{keywordtype}{double} kappa\_param = 5;
\hyperlink{Sacado__example_8cc_a868b94676739e612d9c95940e70892a9}{fad\_double} kappa (kappa\_param);
\end{DoxyCode}
 The second constant remains as a double just to show the difference. 
\begin{DoxyCode}
\textcolor{keywordtype}{double} mu = 2;

SymmetricTensor<2,dim, fad\_double> sigma, eps;
\end{DoxyCode}
 To simplify the access to the dofs we define a map that relate the components of our strain tensor to the dof-\/nbr 
\begin{DoxyCode}
std::map<unsigned int,std::pair<unsigned int,unsigned int>> std\_map\_indicies;
\end{DoxyCode}
 The point at which the derivative shall be computed\+: ~\newline
As mentioned previously, we will implement this example for 2D and 3D, hence we once have to set up a strain tensor and the derivatives for 3D with 6 independent components ... 
\begin{DoxyCode}
\textcolor{keywordflow}{if}(dim==3)
\{
    eps[0][0] = 1;
    eps[1][1] = 2;
    eps[2][2] = 3;

    eps[0][1] = 4;
    eps[0][2] = 5;
    eps[1][2] = 6;


    eps[0][0].diff(0,6);
    eps[0][1].diff(1,6);
    eps[0][2].diff(2,6);
    eps[1][1].diff(3,6);
    eps[1][2].diff(4,6);
    eps[2][2].diff(5,6);
\end{DoxyCode}
 By using the map and the following pairs, we have to set up the relation between strain components and dofs only once and can use the map to access the entries of the list later, without possibly mixing up indices and creating errors. Please don\textquotesingle{}t be confused, but the dofs in the Wrapper are set up in a different order that we showed earlier. Earlier\+: (0,0)-\/(1,1)-\/(2,2)-\/...; Now\+: (0,0)-\/(0,1)-\/(0,2)-\/... 
\begin{DoxyCode}
    std::pair<unsigned int, unsigned int> tmp\_pair;
    tmp\_pair.first=0; tmp\_pair.second=0;
    std\_map\_indicies[0] = tmp\_pair;

    tmp\_pair.first=0; tmp\_pair.second=1;
    std\_map\_indicies[1] = tmp\_pair;

    tmp\_pair.first=0; tmp\_pair.second=2;
    std\_map\_indicies[2] = tmp\_pair;

    tmp\_pair.first=1; tmp\_pair.second=1;
    std\_map\_indicies[3] = tmp\_pair;

    tmp\_pair.first=1; tmp\_pair.second=2;
    std\_map\_indicies[4] = tmp\_pair;

    tmp\_pair.first=2; tmp\_pair.second=2;
    std\_map\_indicies[5] = tmp\_pair;
\}
\end{DoxyCode}
 ... and once for 2D with just 3 independent components. 
\begin{DoxyCode}
\textcolor{keywordflow}{else} \textcolor{keywordflow}{if}(dim==2)
\{
    eps[0][0] = 1;
    eps[1][1] = 2;

    eps[0][1] = 4;


    eps[0][0].diff(0,3);
    eps[0][1].diff(1,3);
    eps[1][1].diff(2,3);

    std::pair<unsigned int, unsigned int> tmp\_pair;
    tmp\_pair.first=0; tmp\_pair.second=0;
    std\_map\_indicies[0] = tmp\_pair;

    tmp\_pair.first=0; tmp\_pair.second=1;
    std\_map\_indicies[1] = tmp\_pair;

    tmp\_pair.first=1; tmp\_pair.second=1;
    std\_map\_indicies[2] = tmp\_pair;        
\}
\textcolor{keywordflow}{else}
\{
    \textcolor{keywordflow}{throw} std::runtime\_error(\textcolor{stringliteral}{"only dim==2 or dim==3 allowed"});
\}
\end{DoxyCode}
 Instead of calling the $\ast$.diff($\ast$) on the components one-\/by-\/one we could also use the following for-\/loop, so we also use the map to set the dofs (as we will do in the Wrapper later). 
\begin{DoxyCode}
\textcolor{keywordflow}{for} ( \textcolor{keywordtype}{unsigned} \textcolor{keywordtype}{int} x=0; x<((dim==2)?3:6); ++x )
\{
    \textcolor{keywordtype}{unsigned} \textcolor{keywordtype}{int} i=std\_map\_indicies[x].first;
    \textcolor{keywordtype}{unsigned} \textcolor{keywordtype}{int} j=std\_map\_indicies[x].second;
    eps[i][j].diff(x,((dim==2)?3:6));
\}
\end{DoxyCode}


For our slightly more complicated stress equation we need the unit and deviatoric tensors. We can simply define them by writing the values of the already existing deal.\+ii functions into newly defined Symmetric\+Tensors build from fad\+\_\+doubles. 
\begin{DoxyCode}
SymmetricTensor<2,dim, fad\_double> stdTensor\_I (( unit\_symmetric\_tensor<dim,fad\_double>()) );
SymmetricTensor<4,dim, fad\_double> stdTensor\_Idev ( (deviator\_tensor<dim,fad\_double>()) );
\end{DoxyCode}
 With everything set and defined, we can compute our stress {\itshape sigma} according to\+: \[ \sigma = \kappa \cdot trace(\varepsilon) \cdot \boldsymbol{I} + 2 \cdot \mu \cdot \varepsilon^{dev} \] Here you can see that we can directly multiply the constant and the tensors when kappa is also declared as fad\+\_\+double 
\begin{DoxyCode}
sigma = kappa * (trace(eps) *  stdTensor\_I);
\end{DoxyCode}
 We didn\textquotesingle{}t do the same for mu to once again emphasize the difference between constants as double and as fad\+\_\+double. ~\newline
The remaining code uses a normal double constant. 
\begin{DoxyCode}
SymmetricTensor<2,dim,fad\_double> tmp = deviator<dim,fad\_double>(symmetrize<dim,fad\_double>(eps)); tmp*=(mu
      *2);
sigma +=  tmp;
\end{DoxyCode}
 The fairly cumbersome computation is caused by the way the operators are set up for tensors out of fad\+\_\+doubles.


\begin{DoxyCode}
std::cout << \textcolor{stringliteral}{"sigma="} << sigma << std::endl;
\end{DoxyCode}
 Now we want to actually build our tangent modulus called {\itshape C\+\_\+\+Sacado} that contains all the derivatives and relates the stress tensor with the strain tensor. ~\newline
The fourth-\/order tensor {\itshape C\+\_\+\+Sacado} is our final goal, we don\textquotesingle{}t have to compute anything that is related to Sacado with this tensor, so we can finally return to our standard Symmetric\+Tensor out of doubles. The latter is necessary to use the tangent in the actual FE code. 
\begin{DoxyCode}
SymmetricTensor<4,dim> C\_Sacado;
\end{DoxyCode}
 As in Ex2 we access the components of the stress tensor one by one. In order to capture all of them we loop over the components i and j of the stress tensor. 
\begin{DoxyCode}
\textcolor{keywordflow}{for} ( \textcolor{keywordtype}{unsigned} \textcolor{keywordtype}{int} i=0; i<dim; ++i)
    \textcolor{keywordflow}{for} ( \textcolor{keywordtype}{unsigned} \textcolor{keywordtype}{int} j=0; j<dim; ++j )
    \{
        \textcolor{keywordtype}{double} *derivs = &sigma[i][j].fastAccessDx(0); \textcolor{comment}{// Access the derivatives of the (i,j)-th component
       of \(\backslash\)a sigma}
\end{DoxyCode}
 To visually ensure that every stress component has in fact all 6 derivatives for 3D or 3 for 2D, we output the size\+: 
\begin{DoxyCode}
std::cout<<\textcolor{stringliteral}{"size: "}<<sigma[i][j].size()<<std::endl;
\end{DoxyCode}
 We loop over all the dofs. To be able to use this independent of the chosen dimension {\itshape dim}, we use a ternary operator to decide whether we have to loop over 6 derivatives or just 3. 
\begin{DoxyCode}
    \textcolor{keywordflow}{for}(\textcolor{keywordtype}{unsigned} \textcolor{keywordtype}{int} x=0;x<((dim==2)?3:6);++x)
    \{
        \textcolor{keywordtype}{unsigned} \textcolor{keywordtype}{int} k=std\_map\_indicies[x].first;
        \textcolor{keywordtype}{unsigned} \textcolor{keywordtype}{int} l=std\_map\_indicies[x].second;

        \textcolor{keywordflow}{if}(k!=l)\textcolor{comment}{/*Compare to Voigt notation since only SymmetricTensor instead of Tensor*/}
        \{
            C\_Sacado[i][j][k][l] = 0.5*derivs[x];
            C\_Sacado[i][j][l][k] = 0.5*derivs[x];
        \}
        \textcolor{keywordflow}{else}
            C\_Sacado[i][j][k][l] = derivs[x];
    \}            

\}
\end{DoxyCode}
 After resembling the fourth-\/order tensor, we now have got our tangent saved in {\itshape C\+\_\+\+Sacado} ready to be used

To ensure that Sacado works properly, we can compute the analytical tangent for comparison 
\begin{DoxyCode}
\textcolor{keywordtype}{double} kappa\_d = 5;
\textcolor{keywordtype}{double} mu\_d = 2;
\end{DoxyCode}
 Our stress equation in this example is still simple enough to derive the tangent analytically by hand\+: \[ \overset{4}{C_{analy}} = \kappa \cdot \boldsymbol{I} \otimes \boldsymbol{I} + 2 \cdot \mu \cdot \overset{4}{I^{dev}} \] 
\begin{DoxyCode}
SymmetricTensor<4,dim> C\_analy = kappa\_d * outer\_product(unit\_symmetric\_tensor<dim>(), 
      unit\_symmetric\_tensor<dim>()) + 2* mu\_d * deviator\_tensor<dim>();
\end{DoxyCode}
 We again define our strain tensor {\itshape eps\+\_\+d} ($\ast$\+\_\+d for standard double in contrast to fad\+\_\+double) 
\begin{DoxyCode}
SymmetricTensor<2,dim> eps\_d;

\textcolor{keywordflow}{if}(dim==3)
\{
    eps\_d[0][0] = 1;
    eps\_d[1][1] = 2;
    eps\_d[2][2] = 3;

    eps\_d[0][1] = 4;
    eps\_d[0][2] = 5;
    eps\_d[2][1] = 6;

\}
\textcolor{keywordflow}{else} \textcolor{keywordflow}{if}(dim==2)
\{
    eps\_d[0][0] = 1;
    eps\_d[1][1] = 2;

    eps\_d[1][0] = 4;

\}
\textcolor{keywordflow}{else}
\{
    \textcolor{keywordflow}{throw} std::runtime\_error(\textcolor{stringliteral}{"only dim==2 or dim==3 allowed"});
\}
\end{DoxyCode}
 \begin{DoxyRefDesc}{Todo}
\item[\hyperlink{todo__todo000005}{Todo}]use boldsymbol for tensors\end{DoxyRefDesc}


To output the stress tensor we first have to compute it. We do this here via \[ \sigma = \overset{4}{C_{analy}} : \varepsilon \] The output exactly matched the result obtained with Sacado. \begin{DoxyNote}{Note}
Checking the Sacado stress tensor against an analytically computed or otherwise determined stress tensor is absolutely no way to check whether the tangent computed via Sacado is correct. When we compute the stress tensor with Sacado and for example mix up a + and -\/ sign, this might not matter at all if the number that is added or subtracted is small. However, for the tangent this nasty sign can be very critical. Just keep in mind\+: the tangent has 81 components and the stress tensor just 9, so how does one want to verify 81 variables by comparing 9?
\end{DoxyNote}

\begin{DoxyCode}
std::cout << \textcolor{stringliteral}{"sigma\_analy: "} << (C\_analy*eps\_d) << std::endl;
\end{DoxyCode}
 That\textquotesingle{}s the reason we compare all the entries in the Sacado and the analytical tensor one by one 
\begin{DoxyCode}
\textcolor{keywordflow}{for} (\textcolor{keywordtype}{unsigned} \textcolor{keywordtype}{int} i=0; i<dim; ++i)
    \textcolor{keywordflow}{for} ( \textcolor{keywordtype}{unsigned} \textcolor{keywordtype}{int} j=0; j<dim; ++j)
        \textcolor{keywordflow}{for} ( \textcolor{keywordtype}{unsigned} \textcolor{keywordtype}{int} k=0; k<dim; ++k)
            \textcolor{keywordflow}{for} ( \textcolor{keywordtype}{unsigned} \textcolor{keywordtype}{int} l=0; l<dim; ++l)
                std::cout << \textcolor{stringliteral}{"C\_analy["}<<i<<\textcolor{stringliteral}{"]["}<<j<<\textcolor{stringliteral}{"]["}<<k<<\textcolor{stringliteral}{"]["}<<l<<\textcolor{stringliteral}{"] = "} << C\_analy[i][j][k][l] << \textcolor{stringliteral}{"
       vs C\_Sacado: "} << C\_Sacado[i][j][k][l] << std::endl;
\end{DoxyCode}
 To simplify the comparison we compute a scalar error as the sum of the absolute differences of each component 
\begin{DoxyCode}
\textcolor{keywordtype}{double} error\_Sacado\_vs\_analy=0;
\textcolor{keywordflow}{for} (\textcolor{keywordtype}{unsigned} \textcolor{keywordtype}{int} i=0; i<dim; ++i)
    \textcolor{keywordflow}{for} ( \textcolor{keywordtype}{unsigned} \textcolor{keywordtype}{int} j=0; j<dim; ++j)
        \textcolor{keywordflow}{for} ( \textcolor{keywordtype}{unsigned} \textcolor{keywordtype}{int} k=0; k<dim; ++k)
            \textcolor{keywordflow}{for} ( \textcolor{keywordtype}{unsigned} \textcolor{keywordtype}{int} l=0; l<dim; ++l)
                error\_Sacado\_vs\_analy += std::fabs(C\_Sacado[i][j][k][l] - C\_analy[i][j][k][l]);
\end{DoxyCode}
 As desired\+: The numerical error is zero (0 in double precision) and the tensor components are equal 
\begin{DoxyCode}
    std::cout << \textcolor{stringliteral}{"numerical error: "} << error\_Sacado\_vs\_analy << std::endl;
\}
\end{DoxyCode}
 \hypertarget{index_Ex3B}{}\section{3\+B. Example\+: Using the wrapper for Ex3}\label{index_Ex3B}

\begin{DoxyCode}
\textcolor{keywordtype}{void} \hyperlink{Sacado__example_8cc_ae63cc8526935cb0512668e83cfc7b929}{sacado\_test\_3B} ()
\{
    std::cout << \textcolor{stringliteral}{"Test 3B:"} << std::endl;
    \textcolor{keyword}{const} \textcolor{keywordtype}{unsigned} \textcolor{keywordtype}{int} dim=3;
\end{DoxyCode}
 The following declarations are usually input arguments. So you receive the strain tensor and the constants out of doubles. 
\begin{DoxyCode}
SymmetricTensor<2,dim> eps\_d;
eps\_d[0][0] = 1;
eps\_d[1][1] = 2;
eps\_d[2][2] = 3;

eps\_d[0][1] = 4;
eps\_d[0][2] = 5;
eps\_d[1][2] = 6;

\textcolor{keywordtype}{double} kappa = 5;
\textcolor{keywordtype}{double} mu = 2;
\end{DoxyCode}
 Now we start working with Sacado\+: ~\newline
When we use the index notation to compute e.\+g. our stress we do not need to declare our constants (here kappa, mu) as fad\+\_\+double.

We declare our strain tensor as the special data type \hyperlink{classSacado__Wrapper_1_1SymTensor}{Sacado\+\_\+\+Wrapper\+::\+Sym\+Tensor} from the file \char`\"{}\+Sacado\+\_\+\+Wrapper.\+h\char`\"{} where this data type was derived from the Symmetric\+Tensor$<$2,dim,fad\+\_\+double$>$. 
\begin{DoxyCode}
\hyperlink{classSacado__Wrapper_1_1SymTensor}{Sacado\_Wrapper::SymTensor<dim>} eps;
\end{DoxyCode}
 Next we initialize our Sacado strain tensor with the values of the inputed double strain tensor\+: 
\begin{DoxyCode}
eps.\hyperlink{classSacado__Wrapper_1_1SymTensor_acbad579d5ead9e96ff46aa15d9b5aef4}{init}(eps\_d);
\end{DoxyCode}
 We define all the entries in the symmetric tensor {\itshape eps} as the dofs. So we can later derive any variable with respect to the strain tensor {\itshape eps}. 
\begin{DoxyCode}
eps.\hyperlink{classSacado__Wrapper_1_1SymTensor_aa9e0fcc9d4e0a4120bedb8ef9b8d7ecb}{set\_dofs}();
\end{DoxyCode}
 Now we declare our output and auxiliary variables as Sacado-\/\+Tensors. 
\begin{DoxyCode}
SymmetricTensor<2,dim,fad\_double> sigma;

SymmetricTensor<2,dim, fad\_double> stdTensor\_I (( unit\_symmetric\_tensor<dim,fad\_double>()) );
\end{DoxyCode}
 Our stress equation is now computed in index notation to simplify the use of the constants and especially the use of the {\itshape deviator}. 
\begin{DoxyCode}
\textcolor{keywordflow}{for} ( \textcolor{keywordtype}{unsigned} \textcolor{keywordtype}{int} i=0; i<dim; ++i)
  \textcolor{keywordflow}{for} ( \textcolor{keywordtype}{unsigned} \textcolor{keywordtype}{int} j=0; j<dim; ++j )
      sigma[i][j] = kappa * trace(eps) *  stdTensor\_I[i][j] + 2. * mu * deviator(eps)[i][j];
\end{DoxyCode}
 Finally we declare our desired tangent as the fourth order tensor {\itshape C\+\_\+\+Sacado} and compute the tangent via the command {\itshape get\+\_\+tangent}. 
\begin{DoxyCode}
SymmetricTensor<4,dim> C\_Sacado;
eps.\hyperlink{classSacado__Wrapper_1_1SymTensor_ab97427c3b5cab279e58607cf431ab262}{get\_tangent}(C\_Sacado, sigma);
\end{DoxyCode}
 We could again compare the herein computed tangent with the analytical tangent from Ex2, but as before the results are fairly boring, because Sacado hits the analytical tangent exactly --- no surprise for such simple equations.

And that\textquotesingle{}s it. By using the Sacado\+\_\+wrapper we can achieve everything from Ex2 (besides the equations) with just four lines of code namely\+:
\begin{DoxyItemize}
\item eps.\+init(eps\+\_\+d); // To initialize the Sacado strain tensor
\item eps.\+set\+\_\+dofs(); // To declare the components of eps as the dofs
\item eps.\+get\+\_\+tangent($\ast$); // To get the tangent 
\begin{DoxyCode}
\}
\end{DoxyCode}
 
\end{DoxyItemize}\hypertarget{index_Ex4}{}\section{4. Example\+: Computing derivatives with respect to a tensor and a scalar}\label{index_Ex4}

\begin{DoxyCode}
\textcolor{keywordtype}{void} \hyperlink{Sacado__example_8cc_a2f4def4563e31d720e07bc7d6363ebe2}{sacado\_test\_4} ()
\{
    std::cout << \textcolor{stringliteral}{"Test 4:"} << std::endl;
    \textcolor{keyword}{const} \textcolor{keywordtype}{unsigned} \textcolor{keywordtype}{int} dim=3;
\end{DoxyCode}
 The following declarations are usually input arguments. So you receive the strain tensor  eps\+\_\+d, the damage variable {\itshape phi} and the constants {\itshape kappa} and {\itshape mu} out of doubles. 
\begin{DoxyCode}
SymmetricTensor<2,dim> eps\_d;
eps\_d[0][0] = 1;
eps\_d[1][1] = 2;
eps\_d[2][2] = 3;

eps\_d[0][1] = 4;
eps\_d[0][2] = 5;
eps\_d[1][2] = 6;

\textcolor{keywordtype}{double} phi\_d = 0.3;
\end{DoxyCode}
 We don\textquotesingle{}t need these constants in the current example. double kappa = 5; double mu = 2;

We set up our strain tensor as in Ex3B. 
\begin{DoxyCode}
\hyperlink{classSacado__Wrapper_1_1SymTensor}{Sacado\_Wrapper::SymTensor<dim>} eps;
\hyperlink{classSacado__Wrapper_1_1SW__double}{Sacado\_Wrapper::SW\_double<dim>} phi;
\end{DoxyCode}
 Initialize the strain tensor and the damage variable 
\begin{DoxyCode}
eps.\hyperlink{classSacado__Wrapper_1_1SymTensor_acbad579d5ead9e96ff46aa15d9b5aef4}{init}(eps\_d);
phi.\hyperlink{classSacado__Wrapper_1_1SW__double_adca799dd92dadebda9aebc91c797682a}{init}(phi\_d);
\end{DoxyCode}
 Set the dofs, where the argument sets the total nbr of dofs (3 or 6 for the sym. tensor and 1 for the double) 
\begin{DoxyCode}
\textcolor{comment}{//    eps.set\_dofs(eps.n\_independent\_components+1/*an additional dof for phi*/);}
\end{DoxyCode}


In order to also compute derivatives with respect to the scalar {\itshape phi}, we add this scalar to our list of derivatives. Because we have already defined 3 or 6 dofs our additional dof will be placed at the end of this list. We set this up with the member variable start\+\_\+index ... 
\begin{DoxyCode}
\textcolor{comment}{//    phi.start\_index=eps.n\_independent\_components;}
\end{DoxyCode}
 and again using the input argument representing the total number of dofs 
\begin{DoxyCode}
\textcolor{comment}{//    phi.set\_dofs(eps.n\_independent\_components+1);}
\end{DoxyCode}
 All of the above 3 lines of code are automatically done by the Do\+Fs\+\_\+summary class. So, to set our dofs we just create an instance and call set\+\_\+dofs with our variables containing the desired dofs. 
\begin{DoxyCode}
\hyperlink{classSacado__Wrapper_1_1DoFs__summary}{Sacado\_Wrapper::DoFs\_summary<dim>} DoFs\_summary;
DoFs\_summary.\hyperlink{classSacado__Wrapper_1_1DoFs__summary_a556293f6e683cb30151d9faadc2cc90d}{set\_dofs}(eps, phi);
\end{DoxyCode}
 Compute the stress tensor and damage variable {\itshape d} (here we just use some arbitrary equations for testing)\+: ~\newline
Let us first declare our output (and auxiliary) variables as Sacado data types. 
\begin{DoxyCode}
SymmetricTensor<2,dim,fad\_double> sigma;
\hyperlink{Sacado__example_8cc_a868b94676739e612d9c95940e70892a9}{fad\_double} d;
\end{DoxyCode}
 \begin{DoxyRefDesc}{Todo}
\item[\hyperlink{todo__todo000006}{Todo}]It would be nice to use the data types from the \hyperlink{namespaceSacado__Wrapper}{Sacado\+\_\+\+Wrapper} for all the Sacado variables. But somehow the operators (multiply$\ast$, ...) seem to cause conflicts again.\end{DoxyRefDesc}


The actual computation in the following scope uses the exact same equation as your normal computation e. g. via the data type double. Hence, you could either directly compute your stress, etc. via the Sacado variables or you define template functions that contain your equations and are either called templated with double or fad\+\_\+double. When using the first option, please consider the computation time that is generally higher for a computation with fad\+\_\+double than with normal doubles (own experience in a special case\+: slower by factor 30). The second option with templates does not suffer these issues. 
\begin{DoxyCode}
\{
\textcolor{keywordflow}{for} ( \textcolor{keywordtype}{unsigned} \textcolor{keywordtype}{int} i=0; i<dim; ++i)
  \textcolor{keywordflow}{for} ( \textcolor{keywordtype}{unsigned} \textcolor{keywordtype}{int} j=0; j<dim; ++j )
      sigma[i][j] = phi * eps[i][j];
\end{DoxyCode}
 To\+Do\+: strangely when phi is a fad\+\_\+double then the multiplication phi $\ast$ eps works directly without having to use the index notation 
\begin{DoxyCode}
std::cout << \textcolor{stringliteral}{"sigma="} << sigma << std::endl;

d = phi*phi + 25 + trace(eps);
std::cout << \textcolor{stringliteral}{"d="} << d << std::endl;
\}
\end{DoxyCode}
 Get the tangents d\+\_\+sigma / d\+\_\+eps\+: Symmetric\+Tensor with respect to Symmetric\+Tensor 
\begin{DoxyCode}
SymmetricTensor<4,dim> C\_Sacado;
eps.\hyperlink{classSacado__Wrapper_1_1SymTensor_ab97427c3b5cab279e58607cf431ab262}{get\_tangent}(C\_Sacado, sigma);
std::cout << \textcolor{stringliteral}{"C\_Sacado="} << C\_Sacado << std::endl;
\end{DoxyCode}
 Compute the analytical tangent\+: 
\begin{DoxyCode}
SymmetricTensor<4,dim> C\_analy;
C\_analy = phi\_d * identity\_tensor<dim>();
std::cout << \textcolor{stringliteral}{"C\_analy ="} << C\_analy << std::endl;
\end{DoxyCode}
 d\+\_\+d / d\+\_\+eps\+: double with respect to Symmetric\+Tensor 
\begin{DoxyCode}
SymmetricTensor<2,dim> d\_d\_d\_eps;
eps.\hyperlink{classSacado__Wrapper_1_1SymTensor_ab97427c3b5cab279e58607cf431ab262}{get\_tangent}(d\_d\_d\_eps, d);
std::cout << \textcolor{stringliteral}{"d\_d\_d\_eps="} << d\_d\_d\_eps << std::endl;
\end{DoxyCode}
 d\+\_\+sigma / d\+\_\+phi\+: Symmetric\+Tensor with respect to double 
\begin{DoxyCode}
SymmetricTensor<2,dim> d\_sigma\_d\_phi;
phi.\hyperlink{classSacado__Wrapper_1_1SW__double_a2e6eca4457eb22b06172bb5749038f1e}{get\_tangent}(d\_sigma\_d\_phi, sigma);
std::cout << \textcolor{stringliteral}{"d\_sigma\_d\_phi="} << d\_sigma\_d\_phi << std::endl;
std::cout << \textcolor{stringliteral}{"sigma = d\_sigma\_d\_phi * phi = "} << d\_sigma\_d\_phi * phi\_d << std::endl;
\end{DoxyCode}
 d\+\_\+d / d\+\_\+phi\+: double with respect to double 
\begin{DoxyCode}
\textcolor{keywordtype}{double} d\_d\_d\_phi;
phi.\hyperlink{classSacado__Wrapper_1_1SW__double_a2e6eca4457eb22b06172bb5749038f1e}{get\_tangent}(d\_d\_d\_phi, d);
std::cout << \textcolor{stringliteral}{"d\_d\_d\_phi="} << d\_d\_d\_phi << std::endl;
\end{DoxyCode}
 And that\textquotesingle{}s it. By using the Sacado\+\_\+wrapper we can compute derivatives with respect to a tensor and a scalar at the same time (besides the equations) in essence with just the following lines of code namely\+:
\begin{DoxyItemize}
\item eps.\+init(eps\+\_\+d); phi.\+init(phi\+\_\+d); // To initialize the Sacado strain tensor and scalar damage variable
\item Do\+Fs\+\_\+summary.\+set\+\_\+dofs(eps, phi); // To declare the components of eps and phi as the dofs
\item eps.\+get\+\_\+tangent($\ast$); // To get tangents with respect to eps
\item phi.\+get\+\_\+tangent($\ast$); // To get tangents with respect to phi 
\begin{DoxyCode}
\}
\end{DoxyCode}
 
\end{DoxyItemize}\hypertarget{index_Ex5}{}\section{5. Example\+: Using a vector-\/valued equation}\label{index_Ex5}

\begin{DoxyCode}
\textcolor{keywordtype}{void} \hyperlink{Sacado__example_8cc_a327dbbb4ea7fc9840c46d149843a44c2}{sacado\_test\_5} ()
\{
    \textcolor{keyword}{const} \textcolor{keywordtype}{unsigned} \textcolor{keywordtype}{int} dim=3;
    std::cout << \textcolor{stringliteral}{"Test 5:"} << std::endl;
    Tensor<1,dim,fad\_double> \hyperlink{CMakeCache_8txt_aac1d6a1710812201527c735f7c6afbaa}{c};
    \hyperlink{Sacado__example_8cc_a868b94676739e612d9c95940e70892a9}{fad\_double} a,b;
    \textcolor{keywordtype}{unsigned} \textcolor{keywordtype}{int} n\_dofs=2;
    a = 1; b = 2;   \textcolor{comment}{// at the point (a,b) = (1,2)}
    a.diff(0,2);  \textcolor{comment}{// Set a to be dof 0, in a 2-dof system.}
    b.diff(1,2);  \textcolor{comment}{// Set b to be dof 1, in a 2-dof system.}
\end{DoxyCode}
 c is now a vector with three components 
\begin{DoxyCode}
c[0] = 2*a+3*b;
c[1] = 4*a+5*b;
c[2] = 6*a+7*b;
\end{DoxyCode}
 Access to the derivatives works as before. 
\begin{DoxyCode}
    \textcolor{keywordflow}{for}(\textcolor{keywordtype}{unsigned} \textcolor{keywordtype}{int} i=0;i<dim;++i)
    \{
        \textcolor{keyword}{const} \hyperlink{Sacado__example_8cc_a868b94676739e612d9c95940e70892a9}{fad\_double} &derivs = c[i]; \textcolor{comment}{// Access derivatives}
        \textcolor{keywordflow}{for}(\textcolor{keywordtype}{unsigned} \textcolor{keywordtype}{int} j=0;j<n\_dofs;++j)
        \{
            std::cout << \textcolor{stringliteral}{"Derivatives at the point ("} << a << \textcolor{stringliteral}{","} << b << \textcolor{stringliteral}{") for "}
            <<i<<\textcolor{stringliteral}{"th component wrt "}<<j<<\textcolor{stringliteral}{"th direction "}<< std::endl;
            std::cout << \textcolor{stringliteral}{"dc\_i/dxj = "} << derivs.fastAccessDx(j) << std::endl;            
        \}
    \}
\}
\end{DoxyCode}
 \hypertarget{index_Ex6}{}\section{6. Example\+: First and second derivatives -\/ Scalar equation}\label{index_Ex6}
The here shown example was copied from \href{https://github.com/trilinos/Trilinos/blob/master/packages/sacado/example/dfad_dfad_example.cpp}{\tt https\+://github.\+com/trilinos/\+Trilinos/blob/master/packages/sacado/example/dfad\+\_\+dfad\+\_\+example.\+cpp} and modified to get a first impression on how we can work with first and second derivatives 
\begin{DoxyCode}
\textcolor{keywordtype}{void} \hyperlink{Sacado__example_8cc_a27450ab52a9d4250e3f5a5f2a3f8f317}{sacado\_test\_6} ()
\{
    std::cout << \textcolor{stringliteral}{"Test 6:"} << std::endl;
\end{DoxyCode}
 Define the variables used in the computation (inputs\+: a, b; output\+: c; auxiliaries\+: $\ast$) as doubles 
\begin{DoxyCode}
\textcolor{keywordtype}{double} a=1;
\textcolor{keywordtype}{double} b=2;
\end{DoxyCode}
 Number of independent variables (scalar a and b) 
\begin{DoxyCode}
\textcolor{keywordtype}{int} num\_dofs = 2;
\end{DoxyCode}
 Define another data type containing even more Sacado data types \begin{DoxyRefDesc}{Todo}
\item[\hyperlink{todo__todo000007}{Todo}]try to merge the fad\+\_\+double data type with this templated data type \end{DoxyRefDesc}

\begin{DoxyCode}
\textcolor{keyword}{typedef} Sacado::Fad::DFad<double> \hyperlink{Sacado__Wrapper_8h_a7e0893207b87dad05c66a34baac8ed2e}{DFadType};
Sacado::Fad::DFad<DFadType> afad(num\_dofs, 0, a);
Sacado::Fad::DFad<DFadType> bfad(num\_dofs, 1, b);
Sacado::Fad::DFad<DFadType> cfad;
\end{DoxyCode}
 Output the variables\+: We se that the values of {\itshape a} and {\itshape b} are set but the derivatives have not yet been fully declared 
\begin{DoxyCode}
std::cout << \textcolor{stringliteral}{"afad="} << afad << std::endl;
std::cout << \textcolor{stringliteral}{"bfad="} << bfad << std::endl;
std::cout << \textcolor{stringliteral}{"cfad="} << cfad << std::endl;
\end{DoxyCode}
 Now we set the \char`\"{}inner\char`\"{} derivatives. 
\begin{DoxyCode}
afad.val() = \hyperlink{Sacado__example_8cc_a868b94676739e612d9c95940e70892a9}{fad\_double}(num\_dofs, 0, a); \textcolor{comment}{// set afad.val() as the first dof and init it with the
       double a}
bfad.val() = \hyperlink{Sacado__example_8cc_a868b94676739e612d9c95940e70892a9}{fad\_double}(num\_dofs, 1, b);
\end{DoxyCode}
 Compute function and derivative with AD 
\begin{DoxyCode}
cfad = 2*afad + std::cos(afad*bfad);
\end{DoxyCode}
 After this, we output the variables again and see that some additional derivatives have been declared. Furthermore, {\itshape cfad} is filled with the values and derivatives 
\begin{DoxyCode}
std::cout << \textcolor{stringliteral}{"afad="} << afad << std::endl;
std::cout << \textcolor{stringliteral}{"bfad="} << bfad << std::endl;
std::cout << \textcolor{stringliteral}{"cfad="} << cfad << std::endl;
\end{DoxyCode}
 Extract value and derivatives 
\begin{DoxyCode}
\textcolor{keywordtype}{double} c\_ad = cfad.val().val();       \textcolor{comment}{// r}
\textcolor{keywordtype}{double} dcda\_ad = cfad.dx(0).val();    \textcolor{comment}{// dr/da}
\textcolor{keywordtype}{double} dcdb\_ad = cfad.dx(1).val();    \textcolor{comment}{// dr/db}
\textcolor{keywordtype}{double} d2cda2\_ad = cfad.dx(0).dx(0);  \textcolor{comment}{// d^2r/da^2}
\textcolor{keywordtype}{double} d2cdadb\_ad = cfad.dx(0).dx(1); \textcolor{comment}{// d^2r/dadb}
\textcolor{keywordtype}{double} d2cdbda\_ad = cfad.dx(1).dx(0); \textcolor{comment}{// d^2r/dbda}
\textcolor{keywordtype}{double} d2cdb2\_ad = cfad.dx(1).dx(1);  \textcolor{comment}{// d^2/db^2}
\end{DoxyCode}
 Now we can print the actual double value of c and some of the derivatives\+: 
\begin{DoxyCode}
     std::cout << \textcolor{stringliteral}{"c\_ad="} << c\_ad << std::endl;
     std::cout << \textcolor{stringliteral}{"Derivatives at the point ("} << a << \textcolor{stringliteral}{","} << b << \textcolor{stringliteral}{")"} << std::endl;
     std::cout << \textcolor{stringliteral}{"dc/da = "} << dcda\_ad << \textcolor{stringliteral}{", dc/db="} << dcdb\_ad << std::endl;
     std::cout << \textcolor{stringliteral}{"d²c/da² = "} << d2cda2\_ad << \textcolor{stringliteral}{", d²c/db²="} << d2cdb2\_ad << std::endl;
     std::cout << \textcolor{stringliteral}{"d²c/dadb = "} << d2cdadb\_ad << \textcolor{stringliteral}{", d²c/dbda="} << d2cdbda\_ad << std::endl;
\}
\end{DoxyCode}
 \hypertarget{index_Ex7}{}\section{7. Example\+: First and second derivatives -\/ Using tensor (\+The full story)}\label{index_Ex7}

\begin{DoxyCode}
\textcolor{keywordtype}{void} \hyperlink{Sacado__example_8cc_a0b694459e5e15c1578d97e637faba8de}{sacado\_test\_7} ()
\{
    \textcolor{keyword}{const} \textcolor{keywordtype}{unsigned} \textcolor{keywordtype}{int} dim=3;
 
    std::cout << \textcolor{stringliteral}{"Test 7:"} << std::endl;
\end{DoxyCode}
 Defining the inputs (material parameters, strain tensor) 
\begin{DoxyCode}
\textcolor{keywordtype}{double} lambda=1;
\textcolor{keywordtype}{double} mu=2;
SymmetricTensor<2,dim, double> eps;

eps[0][0] = 1.;
eps[1][1] = 2.;
eps[2][2] = 3.;

eps[0][1] = 4.;
eps[0][2] = 5.;
eps[1][2] = 6.;
\end{DoxyCode}
 Here we skip the one-\/field example and right away show the equations for a two-\/field problem with {\itshape eps} and {\itshape phi}. 
\begin{DoxyCode}
\textcolor{keywordtype}{double} phi=0.3;
\end{DoxyCode}
 Setup of the map relating the indices (as before) 
\begin{DoxyCode}
std::map<unsigned int,std::pair<unsigned int,unsigned int>> std\_map\_indicies;

std::pair<unsigned int, unsigned int> tmp\_pair;
tmp\_pair.first=0; tmp\_pair.second=0;
std\_map\_indicies[0] = tmp\_pair;

tmp\_pair.first=0; tmp\_pair.second=1;
std\_map\_indicies[1] = tmp\_pair;

tmp\_pair.first=0; tmp\_pair.second=2;
std\_map\_indicies[2] = tmp\_pair;

tmp\_pair.first=1; tmp\_pair.second=1;
std\_map\_indicies[3] = tmp\_pair;

tmp\_pair.first=1; tmp\_pair.second=2;
std\_map\_indicies[4] = tmp\_pair;

tmp\_pair.first=2; tmp\_pair.second=2;
std\_map\_indicies[5] = tmp\_pair;
\end{DoxyCode}
 Number of independent variables (6 for the tensor and 1 for the scalar phi) 
\begin{DoxyCode}
\textcolor{keyword}{const} \textcolor{keywordtype}{unsigned} \textcolor{keywordtype}{int} nbr\_dofs = 6+1;
\end{DoxyCode}
 Declaring the special data types containing all derivatives 
\begin{DoxyCode}
\textcolor{keyword}{typedef} Sacado::Fad::DFad<double> \hyperlink{Sacado__Wrapper_8h_a7e0893207b87dad05c66a34baac8ed2e}{DFadType};
SymmetricTensor<2,dim, Sacado::Fad::DFad<DFadType> > eps\_fad, eps\_fad\_squared;
Sacado::Fad::DFad<DFadType> phi\_fad;
\end{DoxyCode}
 Setting the dofs 
\begin{DoxyCode}
\textcolor{keywordflow}{for} ( \textcolor{keywordtype}{unsigned} \textcolor{keywordtype}{int} x=0; x<6; ++x )
\{
   \textcolor{keywordtype}{unsigned} \textcolor{keywordtype}{int} i=std\_map\_indicies[x].first;
   \textcolor{keywordtype}{unsigned} \textcolor{keywordtype}{int} j=std\_map\_indicies[x].second;
   (eps\_fad[i][j]).diff( x, nbr\_dofs); \textcolor{comment}{// set up the "inner" derivatives}
   (eps\_fad[i][j]).val() = \hyperlink{Sacado__example_8cc_a868b94676739e612d9c95940e70892a9}{fad\_double}(nbr\_dofs, x, eps[i][j]); \textcolor{comment}{// set up the "outer" derivatives}
\}

phi\_fad.diff( 6, nbr\_dofs );
phi\_fad.val() = \hyperlink{Sacado__example_8cc_a868b94676739e612d9c95940e70892a9}{fad\_double}(nbr\_dofs, 6, phi); \textcolor{comment}{// set up the "outer" derivatives}

std::cout << \textcolor{stringliteral}{"eps\_fad="} << eps\_fad << std::endl;
std::cout << \textcolor{stringliteral}{"phi\_fad="} << phi\_fad << std::endl;
\end{DoxyCode}
 Compute eps² = eps\+\_\+ij $\ast$ eps\+\_\+jk in index notation 
\begin{DoxyCode}
\textcolor{keywordflow}{for} ( \textcolor{keywordtype}{unsigned} \textcolor{keywordtype}{int} i=0; i<dim; ++i)
   \textcolor{keywordflow}{for} ( \textcolor{keywordtype}{unsigned} \textcolor{keywordtype}{int} k=0; k<dim; ++k )
       \textcolor{keywordflow}{for} ( \textcolor{keywordtype}{unsigned} \textcolor{keywordtype}{int} j=0; j<dim; ++j )
           \textcolor{keywordflow}{if} ( i>=k )
               eps\_fad\_squared[i][k] += eps\_fad[i][j] * eps\_fad[j][k];
\end{DoxyCode}
 Compute the strain energy density 
\begin{DoxyCode}
Sacado::Fad::DFad<DFadType> energy;
energy = lambda/2. * trace(eps\_fad)*trace(eps\_fad) + mu * trace(eps\_fad\_squared) + 25 * phi\_fad * trace(
      eps\_fad);
\end{DoxyCode}
 Give some insight into the storage of the values and derivatives 
\begin{DoxyCode}
std::cout << \textcolor{stringliteral}{"energy="} << energy << std::endl;
\end{DoxyCode}
 Compute sigma as \[ \frac{\partial \Psi}{\partial \boldsymbol{\varepsilon}} \] 
\begin{DoxyCode}
SymmetricTensor<2,dim> sigma\_Sac;
\textcolor{keywordflow}{for} ( \textcolor{keywordtype}{unsigned} \textcolor{keywordtype}{int} x=0; x<6; ++x )
\{
   \textcolor{keywordtype}{unsigned} \textcolor{keywordtype}{int} i=std\_map\_indicies[x].first;
   \textcolor{keywordtype}{unsigned} \textcolor{keywordtype}{int} j=std\_map\_indicies[x].second;
   \textcolor{keywordflow}{if} ( i!=j )
       sigma\_Sac[i][j] = 0.5 * energy.dx(x).val();
   \textcolor{keywordflow}{else}
       sigma\_Sac[i][j] = energy.dx(x).val();
\}
std::cout << \textcolor{stringliteral}{"sigma\_Sacado="} << sigma\_Sac << std::endl;

\textcolor{keywordtype}{double} d\_energy\_d\_phi = energy.dx(6).val();
std::cout << \textcolor{stringliteral}{"d\_energy\_d\_phi="} << d\_energy\_d\_phi << std::endl;
\end{DoxyCode}
 Analytical stress tensor\+: 
\begin{DoxyCode}
SymmetricTensor<2,dim> sigma;
sigma = lambda*trace(eps)*unit\_symmetric\_tensor<dim>() + 2. * mu * eps;
std::cout << \textcolor{stringliteral}{"analy. sigma="} << sigma << std::endl;
\end{DoxyCode}
 Sacado-\/\+Tangent 
\begin{DoxyCode}
SymmetricTensor<4,dim> C\_Sac;
\textcolor{keywordflow}{for}(\textcolor{keywordtype}{unsigned} \textcolor{keywordtype}{int} x=0;x<6;++x)
   \textcolor{keywordflow}{for}(\textcolor{keywordtype}{unsigned} \textcolor{keywordtype}{int} y=0;y<6;++y)
   \{
       \textcolor{keyword}{const} \textcolor{keywordtype}{unsigned} \textcolor{keywordtype}{int} i=std\_map\_indicies[y].first;
       \textcolor{keyword}{const} \textcolor{keywordtype}{unsigned} \textcolor{keywordtype}{int} j=std\_map\_indicies[y].second;
       \textcolor{keyword}{const} \textcolor{keywordtype}{unsigned} \textcolor{keywordtype}{int} k=std\_map\_indicies[x].first;
       \textcolor{keyword}{const} \textcolor{keywordtype}{unsigned} \textcolor{keywordtype}{int} l=std\_map\_indicies[x].second;

       \textcolor{keywordtype}{double} deriv = energy.dx(x).dx(y); \textcolor{comment}{// Access the derivatives of the (i,j)-th component of \(\backslash\)a sigma}

       \textcolor{keywordflow}{if} ( k!=l && i!=j )
           C\_Sac[i][j][k][l] = 0.25* deriv;
       \textcolor{keywordflow}{else} \textcolor{keywordflow}{if}(k!=l)\textcolor{comment}{/*Compare to Voigt notation since only SymmetricTensor instead of Tensor*/}
       \{
           C\_Sac[i][j][k][l] = 0.5*deriv;
           C\_Sac[i][j][l][k] = 0.5*deriv;
       \}
       \textcolor{keywordflow}{else}
           C\_Sac[i][j][k][l] = deriv;
   \}

\textcolor{keywordtype}{double} d2\_energy\_d\_phi\_2 = energy.dx(6).dx(6);
std::cout << \textcolor{stringliteral}{"d2\_energy\_d\_phi\_2="} << d2\_energy\_d\_phi\_2 << std::endl;

SymmetricTensor<2,dim> d2\_energy\_d\_eps\_d\_phi;

SymmetricTensor<2,dim, Sacado::Fad::DFad<DFadType> > sigma\_Sac\_full;
\textcolor{keywordflow}{for} ( \textcolor{keywordtype}{unsigned} \textcolor{keywordtype}{int} x=0; x<6; ++x )
\{
   \textcolor{keywordtype}{unsigned} \textcolor{keywordtype}{int} i=std\_map\_indicies[x].first;
   \textcolor{keywordtype}{unsigned} \textcolor{keywordtype}{int} j=std\_map\_indicies[x].second;
   \textcolor{keywordflow}{if} ( i!=j )
       sigma\_Sac\_full[i][j] = 0.5 * energy.dx(x);
   \textcolor{keywordflow}{else}
       sigma\_Sac\_full[i][j] = energy.dx(x);
\}

std::cout << \textcolor{stringliteral}{"sigma\_Sac\_full="} << sigma\_Sac\_full << std::endl;
d2\_energy\_d\_eps\_d\_phi[0][0] = sigma\_Sac\_full[0][0].val().dx(6);
d2\_energy\_d\_eps\_d\_phi[1][1] = sigma\_Sac\_full[1][1].val().dx(6);
d2\_energy\_d\_eps\_d\_phi[2][2] = sigma\_Sac\_full[2][2].val().dx(6);
d2\_energy\_d\_eps\_d\_phi[0][1] = sigma\_Sac\_full[0][1].val().dx(6);
d2\_energy\_d\_eps\_d\_phi[0][2] = sigma\_Sac\_full[0][2].val().dx(6);
d2\_energy\_d\_eps\_d\_phi[1][2] = sigma\_Sac\_full[1][2].val().dx(6);

std::cout << \textcolor{stringliteral}{"d2\_energy\_d\_eps\_d\_phi="} << d2\_energy\_d\_eps\_d\_phi << std::endl;

SymmetricTensor<2,dim> d2\_energy\_d\_phi\_d\_eps;
d2\_energy\_d\_phi\_d\_eps[0][0] = energy.dx(6).dx(0);
d2\_energy\_d\_phi\_d\_eps[0][1] = energy.dx(6).dx(1);
d2\_energy\_d\_phi\_d\_eps[0][2] = energy.dx(6).dx(2);
d2\_energy\_d\_phi\_d\_eps[1][1] = energy.dx(6).dx(3);
d2\_energy\_d\_phi\_d\_eps[1][2] = energy.dx(6).dx(4);
d2\_energy\_d\_phi\_d\_eps[2][2] = energy.dx(6).dx(5);
std::cout << \textcolor{stringliteral}{"d2\_energy\_d\_phi\_d\_eps="} << d2\_energy\_d\_phi\_d\_eps << std::endl;
\end{DoxyCode}
 Analytical tangent 
\begin{DoxyCode}
     SymmetricTensor<4,dim> C\_analy;
     C\_analy = lambda * outer\_product(unit\_symmetric\_tensor<dim>(), unit\_symmetric\_tensor<dim>()) + 2. * mu
       * identity\_tensor<dim>();
 
    \textcolor{keywordtype}{double} error\_Sacado\_vs\_analy=0;
    \textcolor{keywordflow}{for} (\textcolor{keywordtype}{unsigned} \textcolor{keywordtype}{int} i=0; i<dim; ++i)
        \textcolor{keywordflow}{for} ( \textcolor{keywordtype}{unsigned} \textcolor{keywordtype}{int} j=0; j<dim; ++j)
            \textcolor{keywordflow}{for} ( \textcolor{keywordtype}{unsigned} \textcolor{keywordtype}{int} k=0; k<dim; ++k)
                \textcolor{keywordflow}{for} ( \textcolor{keywordtype}{unsigned} \textcolor{keywordtype}{int} l=0; l<dim; ++l)
                    error\_Sacado\_vs\_analy += std::fabs(C\_Sac[i][j][k][l] - C\_analy[i][j][k][l]);
 
    std::cout << \textcolor{stringliteral}{"Numerical error="} << error\_Sacado\_vs\_analy << std::endl;
\}
\end{DoxyCode}
 \hypertarget{index_Ex8}{}\section{8. Example\+: First and second derivatives -\/ Using the Wrapper}\label{index_Ex8}

\begin{DoxyCode}
\textcolor{keywordtype}{void} \hyperlink{Sacado__example_8cc_aa7108ff8393b98d66dfef50899d048d9}{sacado\_test\_8} ()
 
\{
    \textcolor{keyword}{const} \textcolor{keywordtype}{unsigned} \textcolor{keywordtype}{int} dim=3;
 
    std::cout << \textcolor{stringliteral}{"Test 8:"} << std::endl;
\end{DoxyCode}
 Defining the inputs (material parameters, strain tensor) 
\begin{DoxyCode}
\textcolor{keywordtype}{double} lambda=1;
\textcolor{keywordtype}{double} mu=2;
SymmetricTensor<2,dim, double> eps;
\textcolor{keywordtype}{double} phi = 0.3;

eps[0][0] = 1.;
eps[1][1] = 2.;
eps[2][2] = 3.;

eps[0][1] = 4.;
eps[0][2] = 5.;
eps[1][2] = 6.;
\end{DoxyCode}
 Declaring the special data types containing all derivatives 
\begin{DoxyCode}
\textcolor{keyword}{typedef} Sacado::Fad::DFad<double> \hyperlink{Sacado__Wrapper_8h_a7e0893207b87dad05c66a34baac8ed2e}{DFadType};
\end{DoxyCode}
 Declare the variables {\itshape eps\+\_\+fad} and {\itshape phi\+\_\+fad} as the special Wrapper data types 
\begin{DoxyCode}
\hyperlink{classSacado__Wrapper_1_1SymTensor2}{Sacado\_Wrapper::SymTensor2<dim>} eps\_fad;
\hyperlink{classSacado__Wrapper_1_1SW__double2}{Sacado\_Wrapper::SW\_double2<dim>} phi\_fad;
\end{DoxyCode}
 Declare the summary data type relating all the dofs and initialising them too 
\begin{DoxyCode}
\hyperlink{classSacado__Wrapper_1_1DoFs__summary}{Sacado\_Wrapper::DoFs\_summary<dim>} DoFs\_summary;
DoFs\_summary.\hyperlink{classSacado__Wrapper_1_1DoFs__summary_ae273d0fa3197118a11d7005523e27d8a}{init\_set\_dofs}(eps\_fad, eps, phi\_fad, phi);
\end{DoxyCode}
 The variables are outputted to give some insight into the storage of the values (derivatives still trivial). 
\begin{DoxyCode}
std::cout << \textcolor{stringliteral}{"eps\_fad="} << eps\_fad << std::endl;
std::cout << \textcolor{stringliteral}{"phi\_fad="} << phi\_fad << std::endl;
\end{DoxyCode}
 Compute eps² = eps\+\_\+ij $\ast$ eps\+\_\+jk in index notation 
\begin{DoxyCode}
SymmetricTensor<2,dim, Sacado::Fad::DFad<DFadType> > eps\_fad\_squared;
\textcolor{keywordflow}{for} ( \textcolor{keywordtype}{unsigned} \textcolor{keywordtype}{int} i=0; i<dim; ++i)
   \textcolor{keywordflow}{for} ( \textcolor{keywordtype}{unsigned} \textcolor{keywordtype}{int} k=0; k<dim; ++k )
       \textcolor{keywordflow}{for} ( \textcolor{keywordtype}{unsigned} \textcolor{keywordtype}{int} j=0; j<dim; ++j )
           \textcolor{keywordflow}{if} ( i>=k )
               eps\_fad\_squared[i][k] += eps\_fad[i][j] * eps\_fad[j][k];
\end{DoxyCode}
 Compute the strain energy density 
\begin{DoxyCode}
Sacado::Fad::DFad<DFadType> energy;
energy = lambda/2. * trace(eps\_fad)*trace(eps\_fad) + mu * trace(eps\_fad\_squared) + 25 * phi\_fad * trace(
      eps\_fad);
\end{DoxyCode}
 The energy is outputted (formatted by hand) to give some insight into the storage of the values and derivatives. ~\newline
energy=399 \mbox{[} 17.\+5 32 40 21.\+5 48 25.\+5 150 \mbox{]} ~\newline
 \mbox{[} 17.\+5 \mbox{[} 5 0 0 1 0 1 25 \mbox{]} 32 \mbox{[} 0 8 0 0 0 0 0 \mbox{]} 40 \mbox{[} 0 0 8 0 0 0 0 \mbox{]} ~\newline
 21.\+5 \mbox{[} 1 0 0 5 0 1 25 \mbox{]} 48 \mbox{[} 0 0 0 0 8 0 0 \mbox{]} 25.\+5 \mbox{[} 1 0 0 1 0 5 25 \mbox{]} ~\newline

\begin{DoxyCode}
\textcolor{comment}{//              150 [ 25 0 0 25 0 25 0 ] ]}

 std::cout << \textcolor{stringliteral}{"energy="} << energy << std::endl;
\end{DoxyCode}
 Compute sigma as \[ \boldsymbol{\sigma} = \frac{\partial \Psi}{\partial \boldsymbol{\varepsilon}} \] 
\begin{DoxyCode}
SymmetricTensor<2,dim> sigma\_Sac;
eps\_fad.get\_tangent(sigma\_Sac, energy);
std::cout << \textcolor{stringliteral}{"sigma\_Sacado="} << sigma\_Sac << std::endl;

\textcolor{keywordtype}{double} d\_energy\_d\_phi;
phi\_fad.\hyperlink{classSacado__Wrapper_1_1SW__double2_ad51ba1e79171d60861b28098dfef903d}{get\_tangent}(d\_energy\_d\_phi, energy);
std::cout << \textcolor{stringliteral}{"d\_energy\_d\_phi="} << d\_energy\_d\_phi << std::endl;
\end{DoxyCode}
 Analytical stress tensor\+: 
\begin{DoxyCode}
SymmetricTensor<2,dim> sigma;
sigma = lambda*trace(eps)*unit\_symmetric\_tensor<dim>() + 2. * mu * eps;
std::cout << \textcolor{stringliteral}{"analy. sigma="} << sigma << std::endl;
\end{DoxyCode}
 Sacado stress tangent (or eps curvature) as \[ \frac{\partial^2 \Psi}{\partial \boldsymbol{\varepsilon}^2} \] 
\begin{DoxyCode}
SymmetricTensor<4,dim> C\_Sac;
eps\_fad.get\_curvature(C\_Sac, energy);
\end{DoxyCode}
 Sacado phi curvature as \[ \frac{\partial^2 \Psi}{\partial \varphi^2} \] 
\begin{DoxyCode}
\textcolor{keywordtype}{double} d2\_energy\_d\_phi\_2;
phi\_fad.\hyperlink{classSacado__Wrapper_1_1SW__double2_a7d3f3a21cd842645af9861bf50308825}{get\_curvature}(d2\_energy\_d\_phi\_2, energy);
std::cout << \textcolor{stringliteral}{"d2\_energy\_d\_phi\_2="} << d2\_energy\_d\_phi\_2 << std::endl;
\end{DoxyCode}
 Sacado derivatives \[ \frac{\partial^2 \Psi}{\partial \boldsymbol{\varepsilon} \partial \varphi} \] 
\begin{DoxyCode}
SymmetricTensor<2,dim> d2\_energy\_d\_eps\_d\_phi;
DoFs\_summary.\hyperlink{classSacado__Wrapper_1_1DoFs__summary_adf29bfda10814ecee9572a4751d34db0}{get\_curvature}(d2\_energy\_d\_eps\_d\_phi, energy, eps\_fad, phi\_fad);
std::cout << \textcolor{stringliteral}{"d2\_energy\_d\_eps\_d\_phi="} << d2\_energy\_d\_eps\_d\_phi << std::endl;
\end{DoxyCode}
 Sacado derivatives \[ \frac{\partial^2 \Psi}{\partial \varphi \partial \boldsymbol{\varepsilon}} \] 
\begin{DoxyCode}
SymmetricTensor<2,dim> d2\_energy\_d\_phi\_d\_eps;
DoFs\_summary.\hyperlink{classSacado__Wrapper_1_1DoFs__summary_adf29bfda10814ecee9572a4751d34db0}{get\_curvature}(d2\_energy\_d\_phi\_d\_eps, energy, phi\_fad, eps\_fad);
std::cout << \textcolor{stringliteral}{"d2\_energy\_d\_phi\_d\_eps="} << d2\_energy\_d\_phi\_d\_eps << std::endl;
\end{DoxyCode}
 When you consider the output\+: ~\newline
d2\+\_\+energy\+\_\+d\+\_\+eps\+\_\+d\+\_\+phi=25 0 0 0 25 0 0 0 25 ~\newline
d2\+\_\+energy\+\_\+d\+\_\+phi\+\_\+d\+\_\+eps=25 0 0 0 25 0 0 0 25 ~\newline
in detail you will notice that both second derivatives are identical. This compplies with the Schwarz integrability condition (Symmetry of second derivatives) (ignoring all limitation and requirements), it holds \[ \frac{\partial^2 \Psi}{\partial \boldsymbol{\varepsilon} \partial \varphi} = \frac{\partial^2 \Psi}{\partial \varphi \partial \boldsymbol{\varepsilon}} \]

Analytical stress tangent 
\begin{DoxyCode}
SymmetricTensor<4,dim> C\_analy;
C\_analy = lambda * outer\_product(unit\_symmetric\_tensor<dim>(), unit\_symmetric\_tensor<dim>()) + 2. * mu * 
      identity\_tensor<dim>();
\end{DoxyCode}
 Compute the error for the stress tangent 
\begin{DoxyCode}
     \textcolor{keywordtype}{double} error\_Sacado\_vs\_analy=0;
     \textcolor{keywordflow}{for} (\textcolor{keywordtype}{unsigned} \textcolor{keywordtype}{int} i=0; i<dim; ++i)
        \textcolor{keywordflow}{for} ( \textcolor{keywordtype}{unsigned} \textcolor{keywordtype}{int} j=0; j<dim; ++j)
            \textcolor{keywordflow}{for} ( \textcolor{keywordtype}{unsigned} \textcolor{keywordtype}{int} k=0; k<dim; ++k)
                \textcolor{keywordflow}{for} ( \textcolor{keywordtype}{unsigned} \textcolor{keywordtype}{int} l=0; l<dim; ++l)
                    error\_Sacado\_vs\_analy += std::fabs(C\_Sac[i][j][k][l] - C\_analy[i][j][k][l]);
     std::cout << \textcolor{stringliteral}{"Numerical error="} << error\_Sacado\_vs\_analy << std::endl;
\}
 
 
 
\textcolor{comment}{/*}
\textcolor{comment}{ * The main function just calls all the examples and puts some space between the outputs.}
\textcolor{comment}{ */}
\textcolor{keywordtype}{int} \hyperlink{Sacado__example_8cc_ae66f6b31b5ad750f1fe042a706a4e3d4}{main} ()
\{
    \hyperlink{Sacado__example_8cc_a71b2675e62203edc430e7ffc8a365193}{sacado\_test\_scalar} ();
 
    std::cout << std::endl;
 
    \hyperlink{Sacado__example_8cc_a8ef4ff1e9526ca8451cdcd1678366d2c}{sacado\_test\_2} ();
 
    std::cout << std::endl;
 
    \hyperlink{Sacado__example_8cc_ae45e1df0eec246dbb6f2c3d28a2a58e4}{sacado\_test\_3} ();
 
    std::cout << std::endl;
 
    \hyperlink{Sacado__example_8cc_ae63cc8526935cb0512668e83cfc7b929}{sacado\_test\_3B} ();
 
    std::cout << std::endl;
 
    \hyperlink{Sacado__example_8cc_a2f4def4563e31d720e07bc7d6363ebe2}{sacado\_test\_4}();
 
    std::cout << std::endl;
 
    \hyperlink{Sacado__example_8cc_a327dbbb4ea7fc9840c46d149843a44c2}{sacado\_test\_5}();
 
    std::cout << std::endl;
 
    \hyperlink{Sacado__example_8cc_a27450ab52a9d4250e3f5a5f2a3f8f317}{sacado\_test\_6}();
 
    std::cout << std::endl;
 
    \hyperlink{Sacado__example_8cc_a0b694459e5e15c1578d97e637faba8de}{sacado\_test\_7}();
 
    std::cout << std::endl;
 
    \hyperlink{Sacado__example_8cc_aa7108ff8393b98d66dfef50899d048d9}{sacado\_test\_8}();
\}
\end{DoxyCode}
\hypertarget{index_END}{}\section{The End}\label{index_END}
Hosted via Git\+Hub according to \href{https://goseeky.wordpress.com/2017/07/22/documentation-101-doxygen-with-github-pages/}{\tt https\+://goseeky.\+wordpress.\+com/2017/07/22/documentation-\/101-\/doxygen-\/with-\/github-\/pages/} 